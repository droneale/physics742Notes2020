Liouville's theorem can apply to both Hamiltonian and non-Hamiltonian systems. In Physics 742, we have focused on its application to Hamiltonian systems.
\begin{tcolorbox}[title=Liouville's Theorem \& Hamiltonian Systems]
A \textbf{Hamiltonian} system in statistical mechanics could describe a closed system.
\\ \\
The flow of a Hamiltonian system preserves phase space volume; for a collection of systems phase space volume is constant in time.
\end{tcolorbox}
\begin{tcolorbox}[title=Liouville's Theorem \& Non-Hamiltonian Systems]
A \textbf{non-Hamiltonian} system in statistical mechanics could describe an open system, e.g. a gas heated by a source (being `driven').
\\ \\
The flow of a non-Hamiltonian system does not preserve phase space volume.
\end{tcolorbox}
We will focus attention on Hamiltonian systems. A microscopic property involves individual particles, e.g. velocity or momentum. A macroscopic property involves a large number of particles as a bulk system, e.g. density, volume, or pressure. Liouville's theorem tells us that flows in phase space are incompressible. This means that an arbitrary volume within phase space can be manipulated in a number of ways by a flow (stretched, twisted, etc.), but the total volume stays the same. This is analogous to fluid mechanics, where this is known as an incompressible fluid. Secondly, it tells us that microcanonical ensembles are time independent. A microcanonical ensemble is a group of systems with the same particle number and volume. Therefore, an ensemble of uniform density will remain uniform over time, or at equilibrium, and the total number of systems within the ensemble is constant. This is a basis of an essential assumption in statistical mechanics: a macrostate is represented by a uniform density of microstates. Lastly, Liouville's theorem tell us there are no attractors in Hamiltonian dynamics. All states are equally probable, and probability densities are constant in time.
\\
\textbf{Juliet}